\documentclass[12pt,spanish]{article}
\usepackage{multicol,caption} 
\usepackage[utf8]{inputenc}  
\usepackage[spanish,english]{babel}
\usepackage{graphicx}
\usepackage{amssymb}
\usepackage{textcomp}
\usepackage{nccmath} 
\usepackage{amsmath, amsthm, amsfonts}
\usepackage{enumerate}
\usepackage{caption}
\usepackage{indentfirst} 
\usepackage{titlesec}
\usepackage{authblk}
\usepackage{hyperref}
\usepackage[none]{hyphenat}
\usepackage{apacite}
\usepackage[right=1in,left=1in,top=1in,bottom=1in,includefoot]{geometry}
\usepackage{fancyhdr}
\usepackage{lastpage}
\usepackage{enumerate}
\usepackage{wrapfig}
\usepackage{empheq}
\usepackage{mdframed}
\usepackage{xcolor}
\usepackage{wrapfig}
\usepackage{cancel}

\newcommand{\V}[1]{\mathbf{#1}}
\newcommand{\Vp}[1]{\mathbf{#1}^{\prime}}
\newcommand{\R}[1]{Eq.\ref*{#1}}
\newcommand{\EL}[2]{\frac{d}{dt}\frac{\partial }{\partial \dot{#1}} #2 - \frac{\partial }{\partial #1}#2}
\newcommand{\dt}{\frac{d}{dt}}
\newcommand{\sumserie}{\sum_{n = 0}^{\infty}}

\newenvironment{problem}[2][Problem]
    { \begin{mdframed} \textbf{#1 #2}\vspace{0.5cm}\\ }
    {  \end{mdframed}}
\newenvironment{result}[1]{\begin{mdframed}\vspace{-0.5cm} #1 }{\end{mdframed}}
%[backgroundcolor=gray!20]
\newcommand*\widefbox[1]{\fbox{\hspace{2em}#1\hspace{2em}}}
\renewcommand{\theenumi}{(\textit{\roman{enumi}})}
\newcommand{\myname}{Andrés F. Cerón M.}
\newcommand{\assignment}{Métodos Matemáticos}
\newcommand{\duedate}{\today}
\newcommand{\Titulo}{Tarea - 06: Series de Laurent.}

\pagestyle{fancy}
\fancypagestyle{plain}
{
  \setlength{\topmargin}{-0.5in}
  \setlength{\headheight}{1.05in}
  \setlength{\headsep}{-0.0in}
  \setlength{\headwidth}{\textwidth}
  \setlength{\footskip}{2.0in}
  \fancyhead[l]{
    {\bf \assignment \hfill \myname} \\
    Departamento de Física Aplicada \hfill \today
  }
}

\pagestyle{myheadings}
\markright{\bf{\assignment}}
\title{\Large \bf \Titulo}
\author{}
\date{}



\begin{document}

\thispagestyle{fancy}
\maketitle
\thispagestyle{plain}
\let\oldthefootnote\thefootnote

\let\thefootnote\oldthefootnote
\setlength{\headheight}{0.65in}
\setlength{\textheight}{8.60in}
\pagestyle{myheadings}
\vspace{-2cm}

\begin{problem}{1}
    Use the following method to derive the integration formula:
    \begin{equation}
        \int_{0}^{\infty} e^{-x^{2}}\cos(2bx)dx = \frac{\sqrt{2}}{\pi}e^{-b^{2}} \quad (b>0)
    \end{equation}
    \begin{enumerate}
        \item Show that the sum of the integrals of $e^{-z^{2}}$ along the lower and upper horizontal legs of the rectangular path in the figure:\\
        \includegraphics[width = 0.4\textwidth]{imgs/P1.png}\newline   
        can be written as:
        \vspace{-0.5cm}
        \begin{equation}
            2\int_{0}^{a}e^{-x^{2}} dx - 2e^{b^{2}}\int_{0}^{a}e^{-x^{2}}\cos(2bx)dx
        \end{equation}
        and that the sum of the integrals along the vertical legs on the right and left can be
written as:
        \begin{equation}
            ie^{-a^{2}}\int_{0}^{b}e^{y^{2}-i2ay}dy - ie^{-a^{2}}\int_{0}^{b}e^{y^{2}+i2ay}dy   
        \end{equation}
        Thus, with the aid of the Cauchy-Goursat theorem, show that:
        \begin{equation}
            \int_{0}^{a}e^{-x^{2}}\cos(2bx)dx = e^{-b^{2}}\int_{0}^{a}e^{-x^{2}}dx + e^{-(a+b^{2})}\int_{0}^{a}e^{y^{2}}\sin(2ay)dy
        \end{equation}

        \item By accepting the fact that:
        \begin{equation}
            \label{eq:P1-1}\int_{0}^{\infty}e^{-x^{2}}dx = \frac{\sqrt{\pi}}{2}
        \end{equation}
        and observing that:
        \begin{equation}
            \left|\int_{0}^{a}e^{y^{2}}\sin(2ay)dy\right| \leq \int_{0}^{a}e^{y^{2}}dy
        \end{equation}
        obtain the desired integration formula by letting a tend to infinity in the equation at the end of part (a).
    \end{enumerate}
\end{problem}
\subsection*{Solución}
El teorema de Cauchy-Goursat nos indica que la integral sobre la función $f$ que es analítica para todos los puntos contenidos en $C$, será cero. Esto hace necesario comprobar que la función $f(z) = e^{-z^{2}}$ es analítica en este contorno, debido a que esta función no tiene ninguna singularidad dentro de $C$ entonces se dice que es analítica en $C$. 

\begin{gather}
    \oint e^{z^{2}}dz = 0
\end{gather}
la parametrización indicada para esta función implica que para los caminos horizontales $z_1 = x + i0 \quad dz_1 = dx$ y $z_2 = x + ib \quad dz_2 = dx$, entonces, para estos caminos la integral será:

\begin{gather*}
    C_1 + C_2 = \int_{-a}^{a}e^{-x^2}dx + \int_{a}^{-a}e^{-(x + ib)^2}dx = 2\int_{0}^{a}e^{-x^2}dx - 2e^{b^2}\int_{0}^{a}e^{-x^2}e^{- i2xb}dx
\end{gather*}
usando el teorema de De Moivre, para expresar exponenciales con potencias imaginarias en términos de senos y cosenos se obtiene.

\begin{gather*}
    C_1 + C_2 = 2\int_{0}^{a}e^{-x^2}dx - 2e^{b^2}\int_{0}^{a}e^{-x^2}\left(\cos(2xb) - i\sin(2xb)\right)dx
\end{gather*}
como la función seno tiene simetría impar, entonces en el intervalo de ($-a,a$) su integral será cero. 
\begin{gather}
    C_1 + C_2 = 2\int_{0}^{a}e^{-x^2}dx - 2e^{b^2}\int_{0}^{a}e^{-x^2}\cos(2xb)dx
\end{gather}
Ahora, para los caminos con orientación vertical, las parametrizaciones nos indican que $z_3 = a + iy, \quad dz_3 = idy$ y $z_4 = -a + iy \quad dz_4 = idy$, remplazando.

\begin{gather}
    C_3 + C_4 = \int_{0}^{b}e^{-(a + iy)^2}idy + \int_{b}^{0}e^{-(-a + iy)^2}idy = e^{-a^2}\int_{0}^{b}e^{y^2 - i2ay}idy - e^{-a^2}\int_{0}^{b}e^{y^2 + i2ay}idy
\end{gather}
El teorema de Cauchy-Goursat nos dice que la suma de estas cuatro integrales debe ser cero

\begin{gather*}
    2\int_{0}^{a}e^{-x^2}dx - 2e^{b^2}\int_{0}^{a}e^{-x^2}\cos(2xb)dx +ie^{-a^2}\int_{0}^{b}e^{y^2 - i2ay}dy - ie^{-a^2}\int_{0}^{b}e^{y^2 + i2ay}dy = 0\\ 
    2\int_{0}^{a}e^{-x^2}dx +ie^{-a^2}\int_{0}^{b}e^{y^2}e^{-i2ay}dy =  2e^{b^2}\int_{0}^{a}e^{-x^2}\cos(2xb)dx  + ie^{-a^2}\int_{0}^{b}e^{y^2}e^{i2ay}dy
\end{gather*}
Expresando los términos exponenciales con potencias imaginarias en términos de funciones trigonométricas, sumando y cancelando términos semejantes se obtiene:

\begin{gather*}
    2e^{b^2}\int_{0}^{a}e^{-x^2}\cos(2xb)dx = 2\int_{0}^{a}e^{-x^2}dx +  2e^{-a^2}\int_{0}^{b}e^{y^2}\sin(2ay)dy  \\
    \int_{0}^{a}e^{-x^2}\cos(2xb)dx = e^{-b^2}\int_{0}^{a}e^{-x^2}dx +  e^{-(a^2 + b^2)}\int_{0}^{b}e^{y^2}\sin(2ay)dy
\end{gather*}
cuando $a \rightarrow \infty$ se obtiene 

\begin{gather*}
    \int_{0}^{\infty}e^{-x^2}\cos(2xb)dx = e^{-b^2}\int_{0}^{\infty}e^{-x^2}dx +  \lim_{a\rightarrow\infty}e^{-(a^2 + b^2)}\int_{0}^{b}e^{y^2}\sin(2ay)dy
\end{gather*}
analizando el segundo término del lado derecho la expresión

\begin{gather*}
    \lim_{a\rightarrow\infty}e^{-(a^2 + b^2)}\int_{0}^{b}e^{y^2}\sin(2ay)dy = \cancelto{0}{\lim_{a\rightarrow\infty}e^{-(a^2 + b^2)}} \quad \left(\lim_{a\rightarrow\infty} \int_{0}^{b} e^{y^2}\sin(2ay)dy\right) = 0
\end{gather*}
la integral del término del lado derecho esta dada por (\ref*{eq:P1-1}), esto implica

\begin{mdframed}
    \vspace{-0.5cm}
    \begin{gather}
        \int_{0}^{\infty}e^{-x^2}\cos(2xb)dx = e^{-b^2}\frac{\sqrt{\pi}}{2}
    \end{gather}
\end{mdframed}
\begin{problem}{2}
    In each case, write the function $f(z)$ in the form $f(z) = u(x, y)+iv(x, y)$:
    \begin{enumerate}
        \item $f(z) = z^3 + z + 1$
        \item $f(z) = \frac{z^{*}}{z}$
    \end{enumerate}
\end{problem}

\subsection*{Solución}

\begin{enumerate}
    \item $f(z) = z^3 + z + 1$
    
    Teniendo en cuenta que $z = x + iy$ entonces 

    \begin{gather*}
        f(z) = (x + iy)^3 + (x + iy) + 1\\
        f(z) = (x + iy)(x^2 + 2ixy + i^2y^2) + (x + iy) + 1\\
        f(z) = (x^3 + 2ix^2y + i^2y^2x) + (iyx^2 + 2i^2xy^2 + y^3i^3) + (x + iy) + 1\\
        f(z) = x^3 + 2ix^2y - y^2x + iyx^2 - 2xy^2 - y^3i + x + iy + 1\\
    \end{gather*}

    Finalmente agrupando los valores reales e imaginarios 

    \begin{gather*}
        f(z) = (x^3 - y^2x - 2xy^2 + x +1) + (2x^2y + yx^2 - y^3  + y)i\\
        f(z) = (x^3 - 3xy^2 + x +1) + (y + 3x^2y - y^3)i
    \end{gather*}

    Entonces se tiene que 

    \begin{mdframed}
        \vspace{-0.25cm}
        \begin{gather}
            u(x,y) = (x^3 - 3xy^2 + x +1)\\
            v(x,y) = (y + 3x^2y - y^3)
        \end{gather}
        \vspace{-0.3cm}
    \end{mdframed}
    \newpage
    \item $f(z) = \frac{z^{*}}{z}$
    
    Expresando en términos de $x,y$ entonces 

    \begin{gather*}
        f(z) = \frac{(x-iy)}{(x+iy)}\\
        f(z) = \frac{x}{(x+iy)} - \frac{iy}{(x+iy)} \\
        f(z) = \frac{x}{(x+iy)}\frac{(x-iy)}{(x-iy)} - \frac{iy}{(x+iy)}\frac{(x-iy)}{(x-iy)} \\
        f(z) = \frac{x(x-iy)}{(x^2 + y^2)} - \frac{iy(x-iy)}{(x^2+y^2)} \\
        f(z) = \frac{x^2-iyx}{(x^2 + y^2)} - \frac{iyx + y^2}{(x^2+y^2)} \\
        f(z) = \frac{x^2-y^2}{(x^2 + y^2)} - i\frac{2xy}{(x^2+y^2)} \\
    \end{gather*}

    Entonces

    \begin{mdframed}
        \vspace{-0.25cm}
        \begin{gather}
            u(x,y) = \frac{x^2-y^2}{(x^2 + y^2)}\\
            v(x,y) = - \frac{2xy}{(x^2+y^2)}
        \end{gather}
        \vspace{-0.3cm}
    \end{mdframed}
    
\end{enumerate}
\begin{problem}{3}
    Use the identity $\sinh(z + \pi i) = - sinh z$, and the fact that $\sinh z$ is periodic with period $2\pi i$ to find the Taylor series for sinh $z$ about the point $z_0 = \pi i$.
\end{problem}

Previamente se demostró la serie de Maclaurin para $\sin(z)$ y esta dada por.

\begin{gather*}
    \sin(z) = \sumserie \frac{(-1)^n z^{2n+1}}{(2n+1)!}
\end{gather*}
Aprovechando la relación con la función hiperbólica, $\sinh(z) = -\sin(iz)$ se obtiene

\begin{gather*}
    \sinh(z)= -i\sin(iz) = -i\sumserie \frac{(-1)^n (iz)^{2n+1}}{(2n+1)!} =  -i\sumserie \frac{iz^{2n+1}}{(2n+1)!}\\
    \sinh(z)=  \sumserie \frac{z^{2n+1}}{(2n+1)!}
\end{gather*}
Ahora, usando la identidad $\sinh(z + \pi i) = -\sinh z$

\begin{gather*}
    \sinh(z + \pi i) = -  \sumserie \frac{z^{2n+1}}{(2n+1)!}
\end{gather*}
Debido a que esta función es periódica cada $2\pi i$ entonces 

\begin{gather}
    \sinh(z + \pi i - 2\pi i) = \sinh(z - \pi i) = -\sinh(z)
\end{gather}
Realizando la sustitución $u = z- \pi i$ entonces 

\begin{gather*}
    \sinh(z) = -\sinh(z - \pi i)\\
    \sinh(u-\pi i) = - \sinh(u)\\
\end{gather*}
Realizando la expansión en series para $\sinh u$ y posteriormente remplazando $z$ se obtiene.

\begin{mdframed}
    \vspace{-0.4cm}
    \begin{gather}
        \sin(z) = - \sumserie \frac{(z- \pi i)^{2n+1}}{(2n+1)!} 
    \end{gather}
\end{mdframed}
\begin{problem}{4}
    Give two Laurent series expansions in powers of z for the function
    \begin{gather}
        f(z) = \frac{1}{z^2(1-z)}
    \end{gather}
    and specify the regions in which those expansions are valid.
\end{problem}
\subsection*{Solución}
Para esta función existen dos singularidades, en $z = 0$ y en $z = 1$. Esto genera dos regiones donde esta función es analítica
\begin{gather*}
    0 < |z| < 1 \quad \text{y} \quad 1 < |z|  < \infty
\end{gather*}
en la primera región $(0 < |z|)$

\begin{gather*}
    f(z) = \frac{1}{z^2} \frac{1}{1-z} = \frac{1}{z^2} \sum_{n=0}^{\infty}z^{n} = \sum_{n = 0}^{\infty} z^{n-2}
\end{gather*}
remplazando $n \rightarrow n+2$

\begin{mdframed}
    \vspace{-0.5cm}
    \begin{gather}
        f(z) = \sum_{n = -2}^{\infty} z^{n} = \frac{1}{z^2} + \frac{1}{z} + \sum_{n = 0}^{\infty} z^{n} \qquad (0 < |z| < 1)
    \end{gather}
\end{mdframed}
en la segunda región se $(1/|z| < 1)$

\begin{gather*}
    f(z) = \frac{1}{z^3}\frac{1}{1/z - 1} = -\frac{1}{z^3}\frac{1}{1-1/z} = -\frac{1}{z^3}\sum_{n=0}^{\infty}\left(\frac{1}{z}\right)^{n}
\end{gather*}
remplazando $n \rightarrow n - 3$

\begin{mdframed}
    \vspace{-0.5cm}
    \begin{gather}
        f(z) = - \sum_{n = 3}^{\infty}\frac{1}{z^{n}}  \qquad (1< |z| < \infty)
    \end{gather}
\end{mdframed}
\newpage
\begin{problem}{5}
    Show that when $0 < |z - 1| < 2$,
    \begin{gather}
        \frac{z}{(z-1)(z-3)} = -3 \sum_{n = 0}^{\infty}\frac{(z-1)^n}{2^{n+2}} - \frac{1}{2(z-1)}
    \end{gather}
\end{problem}

\subsection*{Solución}
Descomponiendo $f(x)$ en fracciones parciales 

\begin{gather*}
    \frac{z}{(z-1)(z-3)} = \frac{A}{(z-1)} + \frac{B}{(z-3)}\\
    z = A(z-3) + B(z-1) = Az - 3A + Bz - B
\end{gather*}
esto implica el siguiente sistema de ecuaciones

\begin{gather*}
    -3A - B = 0 \quad \rightarrow \quad B = -3A\\
    A + B = 1   \quad \rightarrow \quad A = -1/2 \quad B = 3/2
\end{gather*}
entonces 

\begin{gather}
    \frac{z}{(z-1)(z-3)} = -\frac{1}{2(z-1)} + \frac{3}{2(z-3)}
\end{gather}
como estamos en el intervalo $0 < |z - 1| < 2$, la cota inferior indica que $ z > 1$, por lo tanto, la singularidad en $z=1$ no se puede aproximar en series de potencia. La cota superior indica que $|z-1|/2 < 1$.

\begin{gather*}
    f(z) = -\frac{1}{2(z-1)} - \frac{3}{2}\frac{1}{2 + (1-z)} = -\frac{1}{2(z-1)} - \frac{3}{4}\frac{1}{1 - (z-1)/2}
\end{gather*}
finalmente 

\begin{result}
    \begin{gather}
        f(z) = -\frac{1}{2(z-1)} - \frac{3}{2^2}\sum_{n = 0}^{\infty}\left(\frac{z-1}{2}\right)^n\\
        f(z) = -\frac{1}{2(z-1)} - 3\sum_{n = 0}^{\infty}\frac{(z-1)^n}{2^{n+2}}
    \end{gather}
\end{result}

\end{document}
