\documentclass[12pt,spanish]{article}
\usepackage{multicol,caption} 
\usepackage[utf8]{inputenc}  
\usepackage[spanish,english]{babel}
\usepackage{graphicx}
\usepackage{amssymb}
\usepackage{textcomp}
\usepackage{nccmath} 
\usepackage{amsmath, amsthm, amsfonts}
\usepackage{enumerate}
\usepackage{caption}
\usepackage{indentfirst} 
\usepackage{titlesec}
\usepackage{authblk}
\usepackage{hyperref}
\usepackage[none]{hyphenat}
\usepackage{apacite}
\usepackage[right=1in,left=1in,top=1in,bottom=1in,includefoot]{geometry}
\usepackage{fancyhdr}
\usepackage{lastpage}
\usepackage{enumerate}
\usepackage{wrapfig}
\usepackage{empheq}
\usepackage{mdframed}
\usepackage{xcolor}


\newenvironment{problem}[2][Problem]
    { \begin{mdframed}[backgroundcolor=gray!20] \textbf{#1 #2}\vspace{0.5cm}\\ }
    {  \end{mdframed}}

\newcommand*\widefbox[1]{\fbox{\hspace{2em}#1\hspace{2em}}}
\renewcommand{\theenumi}{(\textit{\roman{enumi}})}
\newcommand{\myname}{Andrés F. Cerón M.}
\newcommand{\assignment}{Métodos Matemáticos}
\newcommand{\duedate}{\today}
\newcommand{\Titulo}{Tarea - 02: Funciones analíticas}

\pagestyle{fancy}
\fancypagestyle{plain}
{
  \setlength{\topmargin}{-0.5in}
  \setlength{\headheight}{1.05in}
  \setlength{\headsep}{-0.0in}
  \setlength{\headwidth}{\textwidth}
  \setlength{\footskip}{2.0in}
  \fancyhead[l]{
    {\bf \assignment \hfill \myname} \\
    Departamento de Física Aplicada \hfill \today
  }
}

\pagestyle{myheadings}
\markright{\bf{\assignment}}
\title{\Large \bf \Titulo}
\author{}
\date{}

\newcommand{\V}[1]{\mathbf{#1}}
\newcommand{\Vp}[1]{\mathbf{#1}^{\prime}}
\newcommand{\R}[1]{Eq.\ref*{#1}}
\newcommand{\EL}[2]{\frac{d}{dt}\frac{\partial }{\partial \dot{#1}} #2 - \frac{\partial }{\partial #1}#2}
\newcommand{\dt}{\frac{d}{dt}}

\begin{document}

\thispagestyle{fancy}
\maketitle
\thispagestyle{plain}
\let\oldthefootnote\thefootnote

\let\thefootnote\oldthefootnote
\setlength{\headheight}{0.65in}
\setlength{\textheight}{8.60in}
\pagestyle{myheadings}
\vspace{-2cm}


\begin{problem}{1}
    Demonstrate that when $f$ is analytic throughout the disk $|z - z_0| < R_2$, expansion of the Laurent series reduces to a Taylor series about $z_0$.
\end{problem}
\newpage
\begin{problem}{2}
    Let $C_0$ denote the circle centered at $z_0$ with radius $R$, and use the parametrization
    \begin{gather*}
        z = z_0 + Re^{i\theta}
    \end{gather*}
    to show that 

    \begin{gather*}
        \int_{C_0} (z - z_0)^{n-1} = 
        \left\{\begin{matrix}
            0 & \text{ when} \;\; n = \pm 1,\pm 2, \dots \\
            2\pi i & \text{when} \;\; n = 0          
        \end{matrix}\right.
    \end{gather*}
\end{problem}

\section*{Solución}

La integral sobre una función compleja esta definida como 

\begin{gather*}
    \int_C f(z)dz = \int_C f[z(\theta)]z'(\theta)d\theta
\end{gather*}

Para este caso la parametrización es $z(\theta) = z_0 + Re^{i\theta}$ y su derivada $z'(\theta) = iRe^{i\theta}$, donde $\theta$ esta en el intervalo $-\pi \leq \theta \leq \pi$. Remplazando estos valores se obtiene 

\begin{gather*}
    \int_C (z - z_0)^{n-1} = \int_{-\pi}^{\pi} [z_0 + Re^{i\theta} - z_0]^{n-1} iRe^{i\theta}d\theta\\
    \int_C (z - z_0)^{n-1} = iR^{n}\int_{-\pi}^{\pi} e^{ni\theta} d\theta
\end{gather*}

En el caso en que $n = 0$ 

\begin{mdframed}
    \vspace{-0.25cm}
    \begin{gather}
        \int_C (z - z_0)^{n-1} = i\int_{-\pi}^{\pi} d\theta = 2i\pi
    \end{gather}
    \vspace{-0.3cm}
\end{mdframed}

Ahora para el caso en que $n \neq 0$ 

\begin{gather*}
    \int_C (z - z_0)^{n-1} = iR^{n}\int_{-\pi}^{\pi} e^{ni\theta} d\theta\\
    \int_C (z - z_0)^{n-1} = iR^{n}\frac{1}{in}[e^{ni\pi}- e^{-ni\pi}]
\end{gather*}

\begin{mdframed}
    \vspace{-0.25cm}
    \begin{gather}
    \int_C (z - z_0)^{n-1} = \frac{2i R^{n}\sin(n\pi)}{n} = 0
\end{gather}
    \vspace{-0.3cm}
\end{mdframed}
\newpage
\begin{problem}{3}
    Suppose that $f(z) = x^2 -y^2 -2y +i(2x-2xy)$, where $z = x+iy$. Write $f(z)$ in terms of $z$ and simplify the result.
\end{problem}


Recordando que la parte real de un número complejo puede ser expresada como 

\begin{gather*}
    \Re = \frac{z + z^{*}}{2}
\end{gather*}

y la parte imaginaria 

\begin{gather*}
    \Im = \frac{z - z^{*}}{2i}
\end{gather*}

Ahora, definiendo $z$ como $z = x + iy$ entonces 

\begin{gather*}
    x = \frac{z + z^{*}}{2} \;\;\;\;\; y = \frac{z - z^{*}}{2i}
\end{gather*}

Remplazando en $f(x)$ 


\begin{gather*}
    f(x) = \left(\frac{z + z^{*}}{2}\right)^{2} - \left( \frac{z - z^{*}}{2i}\right)^2 - 2( \frac{z - z^{*}}{2i}) + i\left(2\frac{z + z^{*}}{2} - 2 \frac{z + z^{*}}{2}\frac{z - z^{*}}{2i}\right)\\
\end{gather*}

Expandiendo los productos notables y simplificando términos 

\begin{gather*}
    f(x) = \left(\frac{z^2 +2zz^{*}+ (z^{*})^2}{4}\right) + \left( \frac{z^2 -2zz^{*} + (z^2)^{*}}{4}\right) - \left( \frac{z - z^{*}}{i}\right) + i\left(z + z^{*} +  i\left(\frac{z^2 - (z^2)^{*}}{2}\right)\right)\\
\end{gather*}

Mediante racionalización se obtiene que $1/i = -i$

\begin{gather*}
    f(x) = \left( \frac{z^2  + (z^2)^{*}}{2}\right) + iz - iz^{*} + iz + iz^{*} -  \left(\frac{z^2 - (z^2)^{*}}{2}\right)
\end{gather*}

Finalmente, eliminando términos semejantes se obtiene la expresión solo en términos de $z$

\begin{mdframed}
    \vspace{-0.25cm}
    \begin{gather}
    f(x) =  (z^{2})^{*} + 2iz
    \end{gather}
    \vspace{-0.3cm}
\end{mdframed}


\newpage
\begin{problem}{4}
    Write the function

    $$f(z) = z + \frac{1}{z}$$

    in the form $f(z) = u(r,\theta) + iv(r,\theta)$
\end{problem}


En la forma polar $z = re^{i\theta}$. Remplazando esto en $f(z)$

\begin{gather*}
    f(z) = re^{i\theta} + \frac{1}{re^{i\theta}}\\
    f(z) = re^{i\theta} + \frac{1}{r}e^{-i\theta}
\end{gather*}

Escribiendo los exponenciales en términos de funciones periódicas 

\begin{gather*}
    f(z) = r(\cos\theta + i\sin\theta) + \frac{1}{r}(\cos\theta - i\sin\theta)
\end{gather*}

Usando propiedad distributiva de la multiplicación 

\begin{gather*}
    f(z) = r\cos\theta + ir\sin\theta + \frac{1}{r}\cos\theta - i\frac{1}{r}\sin\theta
\end{gather*}

Finalmente agrupando los valores rales e imaginarios

\begin{mdframed}
    \vspace{-0.25cm}
    \begin{gather}
        f(z) = \left(r + \frac{1}{r}\right)\cos\theta + i\left(r   - \frac{1}{r}\right)\sin\theta
    \end{gather}
    \vspace{-0.3cm}
\end{mdframed}


\newpage
\begin{problem}{5}
    Given $w = f(z) = u + iv$, suppose that the first-order partial derivatives of $u$ and $v$ with respect to $x$ and $y$ exist everywhere in some neighborhood of a given nonzero point $z_0$ and are continuous at $z_0$. Show that:

\begin{gather*}
    u_x = u_r\cos\theta - u_\theta\frac{\sin\theta}{r} \;\;\;\;\;\; u_y = u_r\sin\theta + u_\theta \frac{\cos\theta}{r}
\end{gather*}
\end{problem}

\section*{Solución}

En primer lugar, se supone a $f(z)$ como uan función analítica, es decir, las condiciones de Cauchy-Riemann se cumplirán en su forma cartesiana. 

\vspace{0.5cm}

La función compleja $f(z)$ tiene tanto su representación cartesiana como polar. Para ultima, se define $u$ y $v$ en función de las variables $\theta$ y $r$ de la siguiente forma. 

\begin{gather*}
    x = r\cos\theta\\
    y = r\sin\theta
\end{gather*}

Esto permite expresar la función $f$ como

\begin{gather*}
    f(z) = u(r,\theta) + iv(r,\theta)
\end{gather*}

Derivando parcialmente $u$ respecto a $r$.
\begin{gather*}
    \frac{\partial u}{\partial r} = \frac{\partial u}{\partial x}\frac{\partial x}{\partial r} + \frac{\partial u}{\partial y}\frac{\partial y}{\partial r}\\     
    u_r = u_x\cos\theta + u_y\sin\theta    
\end{gather*}

De forma análoga para $\theta$
\begin{gather*}
    \frac{\partial u}{\partial \theta} = \frac{\partial u}{\partial x}\frac{\partial x}{\partial \theta} + \frac{\partial u}{\partial y}\frac{\partial y}{\partial \theta}\\     
    u_\theta = - u_x r\sin\theta + u_yr\cos\theta    
\end{gather*}

Tomando estas dos expresiones se forma un sistema de dos ecuaciones con dos incógnitas (Solucionando para $u_x$ y $u_y$).

\begin{equation*}
    \left\{\begin{matrix}
        u_r = u_x\cos\theta + u_y\sin\theta\\
        u_\theta = - u_x r\sin\theta + u_yr\cos\theta
    \end{matrix}\right.
\end{equation*}


Para la primera ecuación se despeja $u_x$ y se remplaza en la segunda ecuación 

\begin{gather*}
    u_x = \frac{u_r}{\cos\theta} - \frac{u_y\sin\theta}{\cos\theta}\\
    u_\theta = - \left(\frac{u_r}{\cos\theta} - \frac{u_y\sin\theta}{\cos\theta}\right) r\sin\theta + u_yr\cos\theta\\
    u_\theta =  -\frac{ru_r\sin\theta}{\cos\theta} + \frac{u_yr\sin^2\theta}{\cos\theta}  + u_yr\cos\theta\\
    u_\theta \cos\theta + ru_r\sin\theta = u_yr\sin^2\theta  + u_yr\cos^2\theta\\
    \frac{u_\theta \cos\theta}{r} + u_r\sin\theta = u_y(\sin^2\theta  +\cos^2\theta)\\
\end{gather*}

Entonces el valor de $u_y$ será 

\begin{mdframed}
    \vspace{-0.25cm}
    \begin{gather}
        u_y = u_r\sin\theta + \frac{u_\theta \cos\theta}{r}
    \end{gather}
    \vspace{-0.3cm}
\end{mdframed}

y remplazando en $u_x$

\begin{gather*}
    u_x = \frac{u_r}{\cos\theta} - \left(\frac{u_\theta \cos\theta}{r} + u_r\sin\theta \right)\frac{\sin\theta}{\cos\theta}\\
    u_x = \left(\frac{u_r}{\cos\theta} -  \frac{u_r\sin^2\theta}{\cos\theta}\right) - \frac{u_\theta\sin\theta}{r} \\
    u_x = \frac{u_r(1-\sin^2\theta)}{\cos\theta}  - \frac{u_\theta\sin\theta}{r} \\
    u_x = \frac{u_r(1-\sin^2\theta)}{\cos\theta}  - \frac{u_\theta\sin\theta}{r}
\end{gather*}

Finalmente 

\begin{mdframed}
    \vspace{-0.25cm}
    \begin{gather}
        u_x = u_r\cos\theta  - \frac{u_\theta\sin\theta}{r}
\end{gather}
    \vspace{-0.3cm}
\end{mdframed}
\newpage
\begin{problem}{6}
    With the aid of the polar form of the Cauchy-Riemann equations, derive the alternative form:

    \begin{gather*}
        f^{\prime}(z_0) = -\frac{i}{z_0}(u_\theta + iv_\theta)
    \end{gather*}
\end{problem}

Si realizamos el mismo proceso que el ejercicio anterior pero para $v$ se obtiene el sistema de ecuaciones 

\begin{equation*}
    \left\{\begin{matrix}
        v_r = v_x\cos\theta + v_y\sin\theta\\
        v_\theta = - v_x r\sin\theta + v_yr\cos\theta
    \end{matrix}\right.
\end{equation*}

y las soluciones para este sistema de ecuaciones son análogas a $u$.

\begin{gather*}
    v_x = v_r\cos\theta  - \frac{v_\theta\sin\theta}{r}\\
    v_y = v_r\sin\theta + \frac{v_\theta \cos\theta}{r}
\end{gather*}

Remplazando estos valores en la definición de una derivada compleja. 

\begin{gather*}
    \frac{df}{dz} = u_r\cos\theta  - \frac{u_\theta\sin\theta}{r} + i\left(v_r\cos\theta  - \frac{v_\theta\sin\theta}{r}\right)
\end{gather*}

Para que esta función sera derivable se tienen que cumplir las ecuaciones de Cauchy Riemann en su forma polar, esto nos permite expresar $ru_r = v_\theta$ y  $-rv_r = u_\theta$. Remplazando 

\begin{gather*}
    \frac{df}{dz} = u_r\cos\theta  - \frac{(-rv_r)\sin\theta}{r} + i\left(v_r\cos\theta  - \frac{ru_r\sin\theta}{r}\right)\\
    \frac{df}{dz} = u_r\cos\theta  + v_r\sin\theta + i\left(v_r\cos\theta  - u_r\sin\theta\right)\\
    \frac{df}{dz} = (u_r + iv_r)\cos\theta  - i( u_r+iv_r)\sin\theta\\
    \frac{df}{dz} = e^{-i\theta}(u_r + iv_r)
\end{gather*}

Este es la ecuación para la derivada de una función compleja en función de las derivadas sobre $r$. Ahora si aplicamos nuevamente las condiciones de Cauchy-Riemann se obtiene 


\begin{mdframed}
    \vspace{-0.25cm}
    \begin{gather}
        \frac{df}{dz} = \frac{1}{e^{i\theta}}\left(\frac{v_\theta}{r} - i\frac{u_\theta}{r}\right) = -\frac{i}{re^{i\theta}}\left(u_\theta + iv_\theta\right)
    \end{gather}    
    \vspace{-0.3cm}
\end{mdframed}





\end{document}
