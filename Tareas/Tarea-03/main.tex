\documentclass[12pt,spanish]{article}
\usepackage{multicol,caption} 
\usepackage[utf8]{inputenc}  
\usepackage[spanish,english]{babel}
\usepackage{graphicx}
\usepackage{amssymb}
\usepackage{textcomp}
\usepackage{nccmath} 
\usepackage{amsmath, amsthm, amsfonts}
\usepackage{enumerate}
\usepackage{caption}
\usepackage{indentfirst} 
\usepackage{titlesec}
\usepackage{authblk}
\usepackage{hyperref}
\usepackage[none]{hyphenat}
\usepackage{apacite}
\usepackage[right=1in,left=1in,top=1in,bottom=1in,includefoot]{geometry}
\usepackage{fancyhdr}
\usepackage{lastpage}
\usepackage{enumerate}
\usepackage{wrapfig}
\usepackage{empheq}
\usepackage{mdframed}
\usepackage{xcolor}


\newenvironment{problem}[2][Problem]
    { \begin{mdframed}[backgroundcolor=gray!20] \textbf{#1 #2}\vspace{0.5cm}\\ }
    {  \end{mdframed}}

\newcommand*\widefbox[1]{\fbox{\hspace{2em}#1\hspace{2em}}}
\renewcommand{\theenumi}{(\textit{\roman{enumi}})}
\newcommand{\myname}{Andrés F. Cerón M.}
\newcommand{\assignment}{Métodos Matemáticos}
\newcommand{\duedate}{\today}
\newcommand{\Titulo}{Tarea - 03: Integración en el plano complejo.}

\pagestyle{fancy}
\fancypagestyle{plain}
{
  \setlength{\topmargin}{-0.5in}
  \setlength{\headheight}{1.05in}
  \setlength{\headsep}{-0.0in}
  \setlength{\headwidth}{\textwidth}
  \setlength{\footskip}{2.0in}
  \fancyhead[l]{
    {\bf \assignment \hfill \myname} \\
    Departamento de Física Aplicada \hfill \today
  }
}

\pagestyle{myheadings}
\markright{\bf{\assignment}}
\title{\Large \bf \Titulo}
\author{}
\date{}

\newcommand{\V}[1]{\mathbf{#1}}
\newcommand{\Vp}[1]{\mathbf{#1}^{\prime}}
\newcommand{\R}[1]{Eq.\ref*{#1}}
\newcommand{\EL}[2]{\frac{d}{dt}\frac{\partial }{\partial \dot{#1}} #2 - \frac{\partial }{\partial #1}#2}
\newcommand{\dt}{\frac{d}{dt}}

\begin{document}

\thispagestyle{fancy}
\maketitle
\thispagestyle{plain}
\let\oldthefootnote\thefootnote

\let\thefootnote\oldthefootnote
\setlength{\headheight}{0.65in}
\setlength{\textheight}{8.60in}
\pagestyle{myheadings}
\vspace{-2cm}


\begin{problem}{1}
    Demonstrate that when $f$ is analytic throughout the disk $|z - z_0| < R_2$, expansion of the Laurent series reduces to a Taylor series about $z_0$.
\end{problem}
\newpage
\begin{problem}{2}
    Let $C_0$ denote the circle centered at $z_0$ with radius $R$, and use the parametrization
    \begin{gather*}
        z = z_0 + Re^{i\theta}
    \end{gather*}
    to show that 

    \begin{gather*}
        \int_{C_0} (z - z_0)^{n-1} = 
        \left\{\begin{matrix}
            0 & \text{ when} \;\; n = \pm 1,\pm 2, \dots \\
            2\pi i & \text{when} \;\; n = 0          
        \end{matrix}\right.
    \end{gather*}
\end{problem}

\section*{Solución}

La integral sobre una función compleja esta definida como 

\begin{gather*}
    \int_C f(z)dz = \int_C f[z(\theta)]z'(\theta)d\theta
\end{gather*}

Para este caso la parametrización es $z(\theta) = z_0 + Re^{i\theta}$ y su derivada $z'(\theta) = iRe^{i\theta}$, donde $\theta$ esta en el intervalo $-\pi \leq \theta \leq \pi$. Remplazando estos valores se obtiene 

\begin{gather*}
    \int_C (z - z_0)^{n-1} = \int_{-\pi}^{\pi} [z_0 + Re^{i\theta} - z_0]^{n-1} iRe^{i\theta}d\theta\\
    \int_C (z - z_0)^{n-1} = iR^{n}\int_{-\pi}^{\pi} e^{ni\theta} d\theta
\end{gather*}

En el caso en que $n = 0$ 

\begin{mdframed}
    \vspace{-0.25cm}
    \begin{gather}
        \int_C (z - z_0)^{n-1} = i\int_{-\pi}^{\pi} d\theta = 2i\pi
    \end{gather}
    \vspace{-0.3cm}
\end{mdframed}

Ahora para el caso en que $n \neq 0$ 

\begin{gather*}
    \int_C (z - z_0)^{n-1} = iR^{n}\int_{-\pi}^{\pi} e^{ni\theta} d\theta\\
    \int_C (z - z_0)^{n-1} = iR^{n}\frac{1}{in}[e^{ni\pi}- e^{-ni\pi}]
\end{gather*}

\begin{mdframed}
    \vspace{-0.25cm}
    \begin{gather}
    \int_C (z - z_0)^{n-1} = \frac{2i R^{n}\sin(n\pi)}{n} = 0
\end{gather}
    \vspace{-0.3cm}
\end{mdframed}
\newpage
\begin{problem}{3}
    Suppose that $f(z) = x^2 -y^2 -2y +i(2x-2xy)$, where $z = x+iy$. Write $f(z)$ in terms of $z$ and simplify the result.
\end{problem}


Recordando que la parte real de un número complejo puede ser expresada como 

\begin{gather*}
    \Re = \frac{z + z^{*}}{2}
\end{gather*}

y la parte imaginaria 

\begin{gather*}
    \Im = \frac{z - z^{*}}{2i}
\end{gather*}

Ahora, definiendo $z$ como $z = x + iy$ entonces 

\begin{gather*}
    x = \frac{z + z^{*}}{2} \;\;\;\;\; y = \frac{z - z^{*}}{2i}
\end{gather*}

Remplazando en $f(x)$ 


\begin{gather*}
    f(x) = \left(\frac{z + z^{*}}{2}\right)^{2} - \left( \frac{z - z^{*}}{2i}\right)^2 - 2( \frac{z - z^{*}}{2i}) + i\left(2\frac{z + z^{*}}{2} - 2 \frac{z + z^{*}}{2}\frac{z - z^{*}}{2i}\right)\\
\end{gather*}

Expandiendo los productos notables y simplificando términos 

\begin{gather*}
    f(x) = \left(\frac{z^2 +2zz^{*}+ (z^{*})^2}{4}\right) + \left( \frac{z^2 -2zz^{*} + (z^2)^{*}}{4}\right) - \left( \frac{z - z^{*}}{i}\right) + i\left(z + z^{*} +  i\left(\frac{z^2 - (z^2)^{*}}{2}\right)\right)\\
\end{gather*}

Mediante racionalización se obtiene que $1/i = -i$

\begin{gather*}
    f(x) = \left( \frac{z^2  + (z^2)^{*}}{2}\right) + iz - iz^{*} + iz + iz^{*} -  \left(\frac{z^2 - (z^2)^{*}}{2}\right)
\end{gather*}

Finalmente, eliminando términos semejantes se obtiene la expresión solo en términos de $z$

\begin{mdframed}
    \vspace{-0.25cm}
    \begin{gather}
    f(x) =  (z^{2})^{*} + 2iz
    \end{gather}
    \vspace{-0.3cm}
\end{mdframed}






\end{document}
