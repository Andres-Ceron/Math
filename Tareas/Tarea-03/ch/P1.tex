\begin{problem}{1}
For the functions $f$ and contours $C$ use parametric representations for $C$, or legs of $C$, to evaluate
\begin{gather*}
    \int_C f(z)dz
\end{gather*}

\begin{itemize}
    \item $f(z) = z - 1$  and $C$ is the arc from $z = 0$ to $z = 2$ consisting of 
    \begin{enumerate}
        \item the semicircle $z = 1 + e^{i\theta}$ with $\pi \leq \theta \leq 2\pi$
        \item the segment $z = x$  with $0 \leq x \leq 2$ of the real axis
    \end{enumerate}
    \item $f(z)$ is defined by means of the equations
    \begin{equation*}
        f(z) = \left\{\begin{matrix}
            1 & \text{when} \;\; y < 0\\ 
            4y& \text{when} \;\; y >0
           \end{matrix}\right.
    \end{equation*}
    and $C$ is the arc from $z = -1 - i$ to $z = 1 + i$ along the curve $y = x^3$.
\end{itemize}    
\end{problem}

\section*{Solución}

\begin{itemize}
    \item Para $z = 1$ y $C$ es un arco de $z = 0$ a $z = 2$
    \begin{enumerate}
        \item  El semicírculo $z = 1 + e^{i\theta}$ 
        
        La integral de contorno esta dada por 

        \begin{gather*}
            \int_C f(z)dz = \int_C f[z(\theta)]z'(\theta)d\theta
        \end{gather*}


        Donde $f[z(\theta)] = (1 + e^{i\theta}) - 1$ y $z'(\theta) = ie^{i\theta}$, remplazando estos valores 

        \begin{gather*}
            \int_C f(z)dz = \int_C  [(1 + e^{i\theta}) - 1] ie^{i\theta}d\theta\\
            \int_C f(z)dz = i\int_\pi^{2\pi}  e^{2i\theta}d\theta\\
        \end{gather*}
        Finalmente 
        \begin{mdframed}
            \vspace{-0.25cm}
            \begin{gather}
                \int_C f(z)dz = \frac{i}{2i}\left[  e^{2i\theta}\right]_\pi^{2\pi} = 0
            \end{gather}    
            \vspace{-0.3cm}
        \end{mdframed}

        \item Para este caso $f[z(x)] = (x) - 1$ y $z'(x) = 1$, remplazando estos valores
        
        \begin{gather*}
            \int_C f(z)dz = \int_C  1(x - 1)dx\\
            \int_C f(z)dz = \int_0^2  xdx- \int_0^2dx 
        \end{gather*}

        Solucionando la integral 

        \begin{mdframed}
            \vspace{-0.25cm}
            \begin{gather}
                \int_C f(z)dz = \left[\frac{x^2}{2} - x\right]_0^{2} = 0
            \end{gather}    
            \vspace{-0.3cm}
        \end{mdframed}
    \end{enumerate}

    \item Para este caso el contorno $C$ está compuesto por dos contornos $C_1$ y $C_2$ según la definición de $f(z)$. Entonces en este caso particular 
    
    \begin{gather*}
        \int_C f(z)dz = \int_{C_1} f(z)dz + \int_{C_2}  f(z)dz\\
        \int_C f(z)dz = \int_{C_1} f[z(x)]z'(x)dx + \int_{C_2}  f[z(x)]z'(x)dz
    \end{gather*}

    En coordenadas cartesianas $z$ esta definido como $z = x + iy$, por tanto, para $C_1$ la parametrización será $z = x + ix^3$ en el intervalo $-1 \leq x \leq 0$. En $C_2$ la parametrización es $z = x + ix^3$ en el intervalo $0 \leq x \leq 1$. Ahora como ambas tienen la misma parametrización pero en diferentes intervalos, la derivada paramétrica será la misma $z' = 1 + 3ix^2$ pero en sus respectivos intervalos. 


    \begin{gather*}
        \int_C f(z)dz = \int_{-1}^{0} 1(1 + 3ix^2)dx + \int_0^{1}(4x^3)(1+3ix^2)dx\\
        \int_C f(z)dz = \int_{-1}^{0} dx + 3i\int_{-1}^{0}x^2dx + 4\int_0^{1}x^3dx +   12i\int_0^{1}x^5dx
    \end{gather*}

    \begin{mdframed}
        \vspace{-0.25cm}
        \begin{gather}
            \int_C f(z)dz = \left[x\right]_{-1}^{0} + 3i\left[\frac{x^3}{3}\right]_{-1}^{0} + 4\left[\frac{x^4}{4}\right]_0^{1} +   12i\left[\frac{x^6}{6}\right]_0^{1} = 2 + 3i
        \end{gather}    
        \vspace{-0.3cm}
    \end{mdframed}
    
\end{itemize}