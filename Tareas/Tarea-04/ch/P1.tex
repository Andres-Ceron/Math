\begin{problem}{1}
    Obtain the Maclaurin series representation
    \begin{gather}
        \label{eq:p1}z\cosh(z^2) = \sumserie \frac{z^{4n+1}}{(2n)!}
    \end{gather}
\end{problem}
En esta demostración se tomaran como valida las siguientes expresiones 

\begin{gather}
    \label{eq:csh}\cosh(z) = \cos(iz)\\
    \label{eq:ss} \sin(z) = \sumserie \frac{(-1)^n z^{2n+1}}{(2n+1)!}
\end{gather}
Derivando la expresión (\ref*{eq:ss}) se obtiene

\begin{gather*}
    \cos(z) = \sumserie (-1)^n\frac{2n+1}{(2n+1)!}z^{2n} = \sumserie (-1)^n\frac{2n+1}{(2n+1)(2n)!}z^{2n}
\end{gather*}
\begin{gather}
    \label{eq:sc}\cos(z) = \sumserie \frac{(-1)^nz^{2n}}{(2n)!}
\end{gather}
Con la serie del coseno (\ref*{eq:sc}) y con la relación entre el coseno hiperbólico se obtiene

\begin{gather*}
    \cosh(z) = \cos(iz) = \sumserie \frac{(-1)^n(iz)^{2n}}{(2n)!}\\
    \cosh(z) = \sumserie \frac{(-1)^n(-1)^nz^{2n}}{(2n)!}= \sumserie \frac{z^{2n}}{(2n)!}
\end{gather*}
Finalmente con esta expresión se puede obtener la serie de Maclaurin de (\ref*{eq:p1}).

\begin{mdframed}
    \vspace{-0.5cm}
    \begin{gather}
        z\cosh(z^2) = z\sumserie \frac{(z^2)^{2n}}{(2n)!} = z\sumserie \frac{z^{4n}}{(2n)!} =  \sumserie \frac{z^{4n+1}}{(2n)!}
    \end{gather}    
\end{mdframed}


