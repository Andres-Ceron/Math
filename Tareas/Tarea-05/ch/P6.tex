\begin{problem}{6}
    Derive the expansions:
    \begin{enumerate}
        \item \begin{gather*}
            f(z) =\frac{\sinh z}{z^2} = \frac{1}{z} + \sumserie \frac{z^{2n+1}}{(2n+3)!}
        \end{gather*}
        \item \begin{gather*}
            f(z) =\frac{\sin(z^2)}{z^4} = \frac{1}{z^2} - \frac{z^2}{3!} + \frac{z^6}{5!} - \frac{z^{10}}{7!} + \dots
        \end{gather*}
    \end{enumerate}    
\end{problem}

\begin{enumerate}
    \item Recordando la expresión para la serie de Maclaurin de la función $\sinh z$ 
    \begin{gather*}
        \sinh z = \sumserie \frac{z^{2n+1}}{(2n+1)!}
    \end{gather*}
    Si se multiplica esta función por $1/z^2$ es necesario definir el dominio sobre el cual esta nueva función será analítica. Dado que la singularidad se encuentra en $z = 0$ entonces la función será analítica para el intervalo $0 < |z| < \infty$

    \begin{gather*}
        \frac{\sinh z}{z^2} =  \frac{1}{z^2}\sumserie \frac{z^{2n+1}}{(2n+1)!} = \sumserie \frac{z^{2n-1}}{(2n+1)!}
    \end{gather*}
    Sacando el valor cuando $n = 0$

    \begin{gather*}
        \frac{\sinh z}{z^2} =  \frac{1}{z} + \sum_{n = 1}^{\infty} \frac{z^{2n-1}}{(2n+1)!}
    \end{gather*}
    Para que la sumatoria vuelva a iniciar en $0$ entonces definimos a $n = n + 1$ 

    \begin{mdframed}
        \vspace{-0.4 cm}
        \begin{gather}
            \frac{\sinh z}{z^2} =  \frac{1}{z} + \sum_{n + 1= 1}^{\infty} \frac{z^{2(n+1)-1}}{(2(n+1)+1)!} = \frac{1}{z} + \sumserie \frac{z^{2n+1}}{(2n+3)!} 
        \end{gather}
    \end{mdframed}

    \item Recordando la serie de Maclaurin para la función del $\sin(z)$
    
    \begin{gather*}
        \sin(z) = \sumserie \frac{(-1)^n z^{2n+1}}{(2n+1)!}
    \end{gather*}
    Remplazando $z = z^2$ en esta función se obtiene 

    \begin{gather*}
        \sin(z^2) = \sumserie \frac{(-1)^n z^{4n+2}}{(2n+1)!}
    \end{gather*}
    Finalmente multiplicamos por la función $f(z) = 1/z^4$ la cual es analítica en el intervalo $0< |z| < \infty$, entonces la serie se cumple solo para este dominio. 

    \begin{gather*}
        \frac{\sin(z^2)}{z^4} = \frac{1}{z^4}\sumserie \frac{(-1)^n z^{4n+2}}{(2n+1)!} = \sumserie \frac{(-1)^n z^{4n-2}}{(2n+1)!} 
    \end{gather*}
    Si se expande esta sumatoria se obtiene 

    \begin{mdframed}
        \vspace{-0.4cm}
        \begin{gather}
            \frac{\sin(z^2)}{z^4} = \sumserie \frac{(-1)^n z^{4n-2}}{(2n+1)!} = \frac{1}{z^2} - \frac{z^2}{3!} + \frac{z^6}{5!} - \frac{z^{10}}{7!} + \dots
        \end{gather}
    \end{mdframed}
\end{enumerate}