\begin{problem}{3}
    Use the identity $\sinh(z + \pi i) = - sinh z$, and the fact that $\sinh z$ is periodic with period $2\pi i$ to find the Taylor series for sinh $z$ about the point $z_0 = \pi i$.
\end{problem}

Previamente se demostró la serie de Maclaurin para $\sin(z)$ y esta dada por.

\begin{gather*}
    \sin(z) = \sumserie \frac{(-1)^n z^{2n+1}}{(2n+1)!}
\end{gather*}
Aprovechando la relación con la función hiperbólica, $\sinh(z) = -\sin(iz)$ se obtiene

\begin{gather*}
    \sinh(z)= -i\sin(iz) = -i\sumserie \frac{(-1)^n (iz)^{2n+1}}{(2n+1)!} =  -i\sumserie \frac{iz^{2n+1}}{(2n+1)!}\\
    \sinh(z)=  \sumserie \frac{z^{2n+1}}{(2n+1)!}
\end{gather*}
Ahora, usando la identidad $\sinh(z + \pi i) = -\sinh z$

\begin{gather*}
    \sinh(z + \pi i) = -  \sumserie \frac{z^{2n+1}}{(2n+1)!}
\end{gather*}
Debido a que esta función es periódica cada $2\pi i$ entonces 

\begin{gather}
    \sinh(z + \pi i - 2\pi i) = \sinh(z - \pi i) = -\sinh(z)
\end{gather}
Realizando la sustitución $u = z- \pi i$ entonces 

\begin{gather*}
    \sinh(z) = -\sinh(z - \pi i)\\
    \sinh(u-\pi i) = - \sinh(u)\\
\end{gather*}
Realizando la expansión en series para $\sinh u$ y posteriormente remplazando $z$ se obtiene.

\begin{mdframed}
    \vspace{-0.4cm}
    \begin{gather}
        \sin(z) = - \sumserie \frac{(z- \pi i)^{2n+1}}{(2n+1)!} 
    \end{gather}
\end{mdframed}